\documentclass[12pt]{article}

\usepackage[margin=1.0in]{geometry}
\usepackage{color}
\usepackage{times}
\usepackage[pdftex]{hyperref}

\newcommand{\fillin}[1]{\textcolor{red}{\textsc{#1}}}

\pagestyle{empty}
\setlength{\parindent}{0in}
\setlength{\parskip}{2ex}
\renewcommand{\baselinestretch}{1.0}

\usepackage{graphicx}
\usepackage{caption}

\begin{document}
	
	\begin{title}
		{\Large\bf Homework 4, MATH455: Due Monday, 03/05/2018}
	\end{title}
	
	\author{\bf Your Name: (replace this)}
	
	\maketitle
	{\bf Instructions}:  The homework assignment editing this \LaTeX\ document.  Download the \LaTeX\ source from the class web page and study
	it to learn more about \LaTeX.  Replace the text with appropriate information.  Run ``pdflatex'' on this document.
	
	You will submit this assignment in two parts:
	\begin{enumerate}
		\item Print out the PDF file and bring it to class, and
		\item Send an e-mail to:
		\begin{center}
			gang@math.binghamton.edu
		\end{center}
		\emph{before class} on the due date with two attachments:
		\begin{itemize}
			\item The \LaTeX\ source file, and
			\item The generated PDF document.
		\end{itemize}
	\end{enumerate}
	\newpage
	Please complete the following:
	\begin{enumerate}
		\item Read chapter 3 and finish questions 3.2, 3.4  (on pages 49-50) in this chapter.
		\begin{verbatim}
			data(cheddar)
			cheeseMod = lm(taste~Acetic+H2S+Lactic,cheddar)
			summary(cheeseMod)
			Call:
			lm(formula = taste ~ Acetic + H2S + Lactic, data = cheddar)
			
			Residuals:
			Min      1Q  Median      3Q     Max 
			-17.390  -6.612  -1.009   4.908  25.449 
			
			Coefficients:
			Estimate Std. Error t value Pr(>|t|)   
			(Intercept) -28.8768    19.7354  -1.463  0.15540   
			Acetic        0.3277     4.4598   0.073  0.94198   
			H2S           3.9118     1.2484   3.133  0.00425 **
			Lactic       19.6705     8.6291   2.280  0.03108 * 
			---
			Signif. codes:  0 ‘***’ 0.001 ‘**’ 0.01 ‘*’ 0.05 ‘.’ 0.1 ‘ ’ 1
			
			Residual standard error: 10.13 on 26 degrees of freedom
			Multiple R-squared:  0.6518,	Adjusted R-squared:  0.6116 
			F-statistic: 16.22 on 3 and 26 DF,  p-value: 3.81e-06
		\end{verbatim}
		Accordingly, H2S and Lactic are the two parameters significant at the 5 percent level.\\
		After applying the exponential function to both Acetic and H2S, we get the following results 
		\begin{verbatim}
			> cheeseModP = lm(taste~exp(Acetic)+exp(H2S)+Lactic,cheddar)
			> summary(cheeseModP)
			
			Call:
			lm(formula = taste ~ exp(Acetic) + exp(H2S) + Lactic, data = cheddar)
			
			Residuals:
			Min      1Q  Median      3Q     Max 
			-16.209  -7.266  -1.651   7.385  26.335 
			
			Coefficients:
			Estimate Std. Error t value Pr(>|t|)  
			(Intercept) -1.897e+01  1.127e+01  -1.684   0.1042  
			exp(Acetic)  1.891e-02  1.562e-02   1.210   0.2371  
			exp(H2S)     7.668e-04  4.188e-04   1.831   0.0786 .
			Lactic       2.501e+01  9.062e+00   2.760   0.0105 *
			---
			Signif. codes:  0 ‘***’ 0.001 ‘**’ 0.01 ‘*’ 0.05 ‘.’ 0.1 ‘ ’ 1
			
			Residual standard error: 11.19 on 26 degrees of freedom
			Multiple R-squared:  0.5754,	Adjusted R-squared:  0.5264 
			F-statistic: 11.75 on 3 and 26 DF,  p-value: 4.746e-05
		\end{verbatim}
		Thus, only Lactic remains statistically significant.\\
		We can not operate the f-test on these data sets 
		\begin{verbatim}
			anova(cheeseMod,cheeseModP)Analysis of Variance Table
			
			Model 1: taste ~ Acetic + H2S + Lactic
			Model 2: taste ~ exp(Acetic) + exp(H2S) + Lactic
			Res.Df    RSS Df Sum of Sq F Pr(>F)
			1     26 2668.4                      
			2     26 3253.6  0    -585.2    
		\end{verbatim}
		This is because our degrees of freedom are the same, and thus we are dividing by zero and will be unable to compute anything.\\
		According to our summary, H2S=3.9118, thus for every increase of .01, we increase taste by .039 approximately.
		\begin{verbatim}
			> log(10)
			[1] 2.302585
			> log(10.01)
			[1] 2.303585
			> log(10.01)/log(10)
			[1] 1.000434
		\end{verbatim}
		So about a .04 percent increase given an additive of .01 on the log scale.	
		\begin{verbatim}
			> scores = lm(total~expend+ratio+salary,sat)
			> scoresSZ = lm(total~expend+ratio,sat)
			> scoresNull=lm(total~1,sat)
			> anova(scores,scoresSZ)
			Analysis of Variance Table
			
			Model 1: total ~ expend + ratio + salary
			Model 2: total ~ expend + ratio
			Res.Df    RSS Df Sum of Sq      F  Pr(>F)  
			1     46 216812                              
			2     47 233443 -1    -16631 3.5285 0.06667 .
			---
			Signif. codes:  0 ‘***’ 0.001 ‘**’ 0.01 ‘*’ 0.05 ‘.’ 0.1 ‘ ’ 1
			> anova(scores,scoresNull)
			Analysis of Variance Table
			
			Model 1: total ~ expend + ratio + salary
			Model 2: total ~ 1
			Res.Df    RSS Df Sum of Sq      F  Pr(>F)  
			1     46 216812                              
			2     49 274308 -3    -57496 4.0662 0.01209 *
			---
			Signif. codes:  0 ‘***’ 0.001 ‘**’ 0.01 ‘*’ 0.05 ‘.’ 0.1 ‘ ’ 1
		\end{verbatim}
		Accordingly, it appears that with $H_0:\beta_{salary}=0$, we can not reject that null hypothesis and salary may not be indicative. Meanwhile, all three parameters do seem to have some indication on total score.
		\begin{verbatim}
			> anova(tscores,scores)
			Analysis of Variance Table
			
			Model 1: total ~ expend + ratio + salary + takers
			Model 2: total ~ expend + ratio + salary
			Res.Df    RSS Df Sum of Sq      F    Pr(>F)    
			1     45  48124                                  
			2     46 216812 -1   -168688 157.74 2.607e-16 ***
			---
			Signif. codes:  0 ‘***’ 0.001 ‘**’ 0.01 ‘*’ 0.05 ‘.’ 0.1 ‘ ’ 1
			> summary(tscores)
			
			Call:
			lm(formula = total ~ expend + ratio + salary + takers, data = sat)
			
			Residuals:
			Min      1Q  Median      3Q     Max 
			-90.531 -20.855  -1.746  15.979  66.571 
			
			Coefficients:
			Estimate Std. Error t value Pr(>|t|)    
			(Intercept) 1045.9715    52.8698  19.784  < 2e-16 ***
			expend         4.4626    10.5465   0.423    0.674    
			ratio         -3.6242     3.2154  -1.127    0.266    
			salary         1.6379     2.3872   0.686    0.496    
			takers        -2.9045     0.2313 -12.559 2.61e-16 ***
			---
			Signif. codes:  0 ‘***’ 0.001 ‘**’ 0.01 ‘*’ 0.05 ‘.’ 0.1 ‘ ’ 1
			
			Residual standard error: 32.7 on 45 degrees of freedom
			Multiple R-squared:  0.8246,	Adjusted R-squared:  0.809 
			F-statistic: 52.88 on 4 and 45 DF,  p-value: < 2.2e-16
		\end{verbatim}
		
		as we can see, the t-value demonstrated in summary is the same as the F value provided by anova, which demonstrates their equivalence
		\item Read chapter 4 and finish questions 4.1, 4.5  (on pages 56-58) in this chapter.
		\item Read chapter 7.3 and finish question 7.8 (on page 111) in this chapter.
		
		
		
	\end{enumerate}
	
\end{document}
